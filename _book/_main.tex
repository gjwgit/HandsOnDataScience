\documentclass[]{book}
\usepackage{lmodern}
\usepackage{amssymb,amsmath}
\usepackage{ifxetex,ifluatex}
\usepackage{fixltx2e} % provides \textsubscript
\ifnum 0\ifxetex 1\fi\ifluatex 1\fi=0 % if pdftex
  \usepackage[T1]{fontenc}
  \usepackage[utf8]{inputenc}
\else % if luatex or xelatex
  \ifxetex
    \usepackage{mathspec}
    \usepackage{xltxtra,xunicode}
  \else
    \usepackage{fontspec}
  \fi
  \defaultfontfeatures{Mapping=tex-text,Scale=MatchLowercase}
  \newcommand{\euro}{€}
\fi
% use upquote if available, for straight quotes in verbatim environments
\IfFileExists{upquote.sty}{\usepackage{upquote}}{}
% use microtype if available
\IfFileExists{microtype.sty}{%
\usepackage{microtype}
\UseMicrotypeSet[protrusion]{basicmath} % disable protrusion for tt fonts
}{}
\usepackage[margin=1in]{geometry}
\usepackage{color}
\usepackage{fancyvrb}
\newcommand{\VerbBar}{|}
\newcommand{\VERB}{\Verb[commandchars=\\\{\}]}
\DefineVerbatimEnvironment{Highlighting}{Verbatim}{commandchars=\\\{\}}
% Add ',fontsize=\small' for more characters per line
\usepackage{framed}
\definecolor{shadecolor}{RGB}{248,248,248}
\newenvironment{Shaded}{\begin{snugshade}}{\end{snugshade}}
\newcommand{\KeywordTok}[1]{\textcolor[rgb]{0.13,0.29,0.53}{\textbf{{#1}}}}
\newcommand{\DataTypeTok}[1]{\textcolor[rgb]{0.13,0.29,0.53}{{#1}}}
\newcommand{\DecValTok}[1]{\textcolor[rgb]{0.00,0.00,0.81}{{#1}}}
\newcommand{\BaseNTok}[1]{\textcolor[rgb]{0.00,0.00,0.81}{{#1}}}
\newcommand{\FloatTok}[1]{\textcolor[rgb]{0.00,0.00,0.81}{{#1}}}
\newcommand{\CharTok}[1]{\textcolor[rgb]{0.31,0.60,0.02}{{#1}}}
\newcommand{\StringTok}[1]{\textcolor[rgb]{0.31,0.60,0.02}{{#1}}}
\newcommand{\CommentTok}[1]{\textcolor[rgb]{0.56,0.35,0.01}{\textit{{#1}}}}
\newcommand{\OtherTok}[1]{\textcolor[rgb]{0.56,0.35,0.01}{{#1}}}
\newcommand{\AlertTok}[1]{\textcolor[rgb]{0.94,0.16,0.16}{{#1}}}
\newcommand{\FunctionTok}[1]{\textcolor[rgb]{0.00,0.00,0.00}{{#1}}}
\newcommand{\RegionMarkerTok}[1]{{#1}}
\newcommand{\ErrorTok}[1]{\textbf{{#1}}}
\newcommand{\NormalTok}[1]{{#1}}
\usepackage{graphicx}
\makeatletter
\def\maxwidth{\ifdim\Gin@nat@width>\linewidth\linewidth\else\Gin@nat@width\fi}
\def\maxheight{\ifdim\Gin@nat@height>\textheight\textheight\else\Gin@nat@height\fi}
\makeatother
% Scale images if necessary, so that they will not overflow the page
% margins by default, and it is still possible to overwrite the defaults
% using explicit options in \includegraphics[width, height, ...]{}
\setkeys{Gin}{width=\maxwidth,height=\maxheight,keepaspectratio}
\ifxetex
  \usepackage[setpagesize=false, % page size defined by xetex
              unicode=false, % unicode breaks when used with xetex
              xetex]{hyperref}
\else
  \usepackage[unicode=true]{hyperref}
\fi
\hypersetup{breaklinks=true,
            bookmarks=true,
            pdfauthor={Graham Williams; Karthik Thirumalai},
            pdftitle={Hands on Data Science with R},
            colorlinks=true,
            citecolor=blue,
            urlcolor=blue,
            linkcolor=magenta,
            pdfborder={0 0 0}}
\urlstyle{same}  % don't use monospace font for urls
\setlength{\parindent}{0pt}
\setlength{\parskip}{6pt plus 2pt minus 1pt}
\setlength{\emergencystretch}{3em}  % prevent overfull lines
\setcounter{secnumdepth}{5}

%%% Use protect on footnotes to avoid problems with footnotes in titles
\let\rmarkdownfootnote\footnote%
\def\footnote{\protect\rmarkdownfootnote}

%%% Change title format to be more compact
\usepackage{titling}

% Create subtitle command for use in maketitle
\newcommand{\subtitle}[1]{
  \posttitle{
    \begin{center}\large#1\end{center}
    }
}

\setlength{\droptitle}{-2em}
  \title{Hands on Data Science with R}
  \pretitle{\vspace{\droptitle}\centering\huge}
  \posttitle{\par}
  \author{Graham Williams \\ Karthik Thirumalai}
  \preauthor{\centering\large\emph}
  \postauthor{\par}
  \date{}
  \predate{}\postdate{}



\begin{document}

\maketitle


{
\hypersetup{linkcolor=black}
\setcounter{tocdepth}{2}
\tableofcontents
}
\section{Introduction}\label{introduction}

Data Science is a forest of skills to learn and master. Hands on Data
Science with R is a survival guide to data science using R. This book
will teach you the practical skills and best practices of data science -
With comprehensive and carefully curated solutions to most data science
problems, this book provides you a swiss army knife to cut throught the
forest.

\begin{center}\includegraphics[width=0.33\linewidth]{graphics/book-cover-A4} \end{center}

\section{Strings Manipulation}\label{strings-manipulation}

In this module we introduce to tools available in R for handling and
processing strings. The required packages for this module include:

\begin{Shaded}
\begin{Highlighting}[]
\KeywordTok{library}\NormalTok{(rattle) }\CommentTok{# The weather dataset.}
\KeywordTok{library}\NormalTok{(stringr) }\CommentTok{# Pre-eminent package for string handling.}
\end{Highlighting}
\end{Shaded}

\subsection{String Concatenation}\label{string-concatenation}

Let us start with the simplest of string operations - concatenation two
strings. The \textbf{cat()} function concatenates objects and could also
print them to screen or to a file. By default it converts even numeric
and other complex objects into character type and then concatenates .
Alternatively we can use the \textbf{paste()} function to concatenate
and print the values to screen. The \textbf{str\_c()} function is
similar to the paste() function but the default separator is white space
and it ignores NULL characters.

\begin{Shaded}
\begin{Highlighting}[]
\KeywordTok{cat}\NormalTok{(}\StringTok{"hello"}\NormalTok{,}\StringTok{"world"}\NormalTok{,}\DataTypeTok{sep=}\StringTok{"}\CharTok{\textbackslash{}t}\StringTok{"}\NormalTok{)}
\end{Highlighting}
\end{Shaded}

\begin{verbatim}
## hello    world
\end{verbatim}

\begin{Shaded}
\begin{Highlighting}[]
\NormalTok{x <-}\StringTok{ }\DecValTok{123} \CommentTok{#Using numeric values with cat}
\KeywordTok{cat} \NormalTok{(}\StringTok{"hello"}\NormalTok{,x,}\DataTypeTok{sep=}\StringTok{"}\CharTok{\textbackslash{}t}\StringTok{"}\NormalTok{)}
\end{Highlighting}
\end{Shaded}

\begin{verbatim}
## hello    123
\end{verbatim}

\begin{Shaded}
\begin{Highlighting}[]
\KeywordTok{paste}\NormalTok{(}\StringTok{"hello"}\NormalTok{,}\StringTok{"world"}\NormalTok{, }\DataTypeTok{sep=}\StringTok{"}\CharTok{\textbackslash{}t}\StringTok{"}\NormalTok{) }\CommentTok{#usage paste function}
\end{Highlighting}
\end{Shaded}

\begin{verbatim}
## [1] "hello\tworld"
\end{verbatim}

\begin{Shaded}
\begin{Highlighting}[]
\KeywordTok{str_c}\NormalTok{(}\StringTok{'hello'}\NormalTok{,}\OtherTok{NULL}\NormalTok{,}\StringTok{'world'}\NormalTok{) }\CommentTok{# str_c with null characters}
\end{Highlighting}
\end{Shaded}

\begin{verbatim}
## [1] "helloworld"
\end{verbatim}

\subsection{String Length}\label{string-length}

The \textbf{nchar()} function in the base package is used to measure the
typical length of the string. The \textbf{str\_length()} package could
also be used to measure string lengths. In comparisson the str\_length()
package handles NA characters more accurately as nchar(NA) returns 2
while str\_length() returns NA. The other advantage of str\_length()
over nchar() is its ability to handle factors robustly.

\begin{Shaded}
\begin{Highlighting}[]
\KeywordTok{nchar}\NormalTok{(}\StringTok{'hello world'}\NormalTok{) }\CommentTok{#nchar functionality}
\end{Highlighting}
\end{Shaded}

\begin{verbatim}
## [1] 11
\end{verbatim}

\begin{Shaded}
\begin{Highlighting}[]
\KeywordTok{nchar}\NormalTok{(}\OtherTok{NA}\NormalTok{)}
\end{Highlighting}
\end{Shaded}

\begin{verbatim}
## [1] NA
\end{verbatim}

\begin{Shaded}
\begin{Highlighting}[]
\KeywordTok{str_length}\NormalTok{(}\StringTok{'hello world'}\NormalTok{) }\CommentTok{#str_length from stringr package}
\end{Highlighting}
\end{Shaded}

\begin{verbatim}
## [1] 11
\end{verbatim}

\begin{Shaded}
\begin{Highlighting}[]
\KeywordTok{str_length}\NormalTok{(}\OtherTok{NA}\NormalTok{)}
\end{Highlighting}
\end{Shaded}

\begin{verbatim}
## [1] NA
\end{verbatim}

\begin{Shaded}
\begin{Highlighting}[]
\NormalTok{factor_example =}\StringTok{ }\KeywordTok{factor}\NormalTok{(}\KeywordTok{c}\NormalTok{(}\DecValTok{1}\NormalTok{, }\DecValTok{1}\NormalTok{, }\DecValTok{0}\NormalTok{,}\DecValTok{0}\NormalTok{), }\DataTypeTok{labels =} \KeywordTok{c}\NormalTok{(}\StringTok{"success"}\NormalTok{, }\StringTok{"fail"}\NormalTok{))}

\CommentTok{#Handling factors str_length}
\KeywordTok{str_length}\NormalTok{(factor_example)}
\end{Highlighting}
\end{Shaded}

\begin{verbatim}
## [1] 4 4 7 7
\end{verbatim}

\begin{Shaded}
\begin{Highlighting}[]
\CommentTok{#Handling factors nchar}
\KeywordTok{nchar}\NormalTok{(factor_example)}
\end{Highlighting}
\end{Shaded}

\begin{verbatim}
## Error in nchar(factor_example): 'nchar()' requires a character vector
\end{verbatim}

\subsection{Case Conversion}\label{case-conversion}

Often during data transformations strings have to be converted from one
case to the other. These simple transformations could be achieved by
\textbf{tolower()} and \textbf{toupper()} functions. The
\textbf{casefolding()} function could also be used as a wrapper to the
two functions.

\begin{Shaded}
\begin{Highlighting}[]
\CommentTok{#Conversion to upper case}
\KeywordTok{toupper}\NormalTok{(}\StringTok{'string manipulation'}\NormalTok{)}
\end{Highlighting}
\end{Shaded}

\begin{verbatim}
## [1] "STRING MANIPULATION"
\end{verbatim}

\begin{Shaded}
\begin{Highlighting}[]
\CommentTok{#Conversion to lower case}
\KeywordTok{tolower}\NormalTok{(}\StringTok{'STRING MANIPULATION'}\NormalTok{)}
\end{Highlighting}
\end{Shaded}

\begin{verbatim}
## [1] "string manipulation"
\end{verbatim}

\begin{Shaded}
\begin{Highlighting}[]
\CommentTok{#casefold to upper}
\KeywordTok{casefold}\NormalTok{(}\StringTok{'string manipulation'}\NormalTok{,}\DataTypeTok{upper=}\OtherTok{TRUE}\NormalTok{)}
\end{Highlighting}
\end{Shaded}

\begin{verbatim}
## [1] "STRING MANIPULATION"
\end{verbatim}

\subsection{Substring Operation}\label{substring-operation}

Finding substrings are one of the most common string manipulation
operations. The \textbf{substr()} could be used to extract, replace
parts of the string. The \textbf{substring()} functions performs the
same operations on a character vector.

\begin{Shaded}
\begin{Highlighting}[]
\CommentTok{#Exctraction strings}
\KeywordTok{substr}\NormalTok{(}\StringTok{'string manipulation'}\NormalTok{,}\DecValTok{3}\NormalTok{,}\DecValTok{6}\NormalTok{)}
\end{Highlighting}
\end{Shaded}

\begin{verbatim}
## [1] "ring"
\end{verbatim}

\begin{Shaded}
\begin{Highlighting}[]
\CommentTok{#Replacing strings with substr}
\NormalTok{s <-}\StringTok{ 'string manipulation'}
\KeywordTok{substr}\NormalTok{(s,}\DecValTok{3}\NormalTok{,}\DecValTok{6}\NormalTok{) <-}\StringTok{ 'RING'}
\NormalTok{s}
\end{Highlighting}
\end{Shaded}

\begin{verbatim}
## [1] "stRING manipulation"
\end{verbatim}

\begin{Shaded}
\begin{Highlighting}[]
\CommentTok{#Extraction from character vectors using substring}
\NormalTok{x <-}\StringTok{ }\KeywordTok{c}\NormalTok{(}\StringTok{'abcd'}\NormalTok{,}\StringTok{'aabcb'}\NormalTok{,}\StringTok{'babcc'}\NormalTok{,}\StringTok{'cabcd'}\NormalTok{)}
\KeywordTok{substring}\NormalTok{(x,}\DecValTok{2}\NormalTok{,}\DataTypeTok{last =} \DecValTok{4}\NormalTok{)}
\end{Highlighting}
\end{Shaded}

\begin{verbatim}
## [1] "bcd" "abc" "abc" "abc"
\end{verbatim}

\begin{Shaded}
\begin{Highlighting}[]
\CommentTok{#Replacing in character vector using substring}
\KeywordTok{substring}\NormalTok{(x,}\DecValTok{2}\NormalTok{,}\DataTypeTok{last=}\DecValTok{4}\NormalTok{) <-}\StringTok{ 'AB'}
\NormalTok{x}
\end{Highlighting}
\end{Shaded}

\begin{verbatim}
## [1] "aABd"  "aABcb" "bABcc" "cABcd"
\end{verbatim}

The stringr package offers \textbf{str\_sub()} which is a equivalent of
substring(). The str\_sub() function handles negative values even more
robustly than the substring() function.

\begin{Shaded}
\begin{Highlighting}[]
\NormalTok{y =}\StringTok{ }\KeywordTok{c}\NormalTok{(}\StringTok{"string"}\NormalTok{, }\StringTok{"manipulation"}\NormalTok{, }\StringTok{"always"}\NormalTok{, }\StringTok{"fascinating"}\NormalTok{)}

\CommentTok{# substring function using negative indices}
\KeywordTok{substring}\NormalTok{(y,}\DataTypeTok{first =} \NormalTok{-}\DecValTok{4}\NormalTok{,}\DataTypeTok{last =} \NormalTok{-}\DecValTok{1}\NormalTok{)}
\end{Highlighting}
\end{Shaded}

\begin{verbatim}
## [1] "" "" "" ""
\end{verbatim}

\begin{Shaded}
\begin{Highlighting}[]
\CommentTok{# str_sub handles negative indices}
\KeywordTok{str_sub}\NormalTok{(y , }\DataTypeTok{start =} \NormalTok{-}\DecValTok{4}\NormalTok{, }\DataTypeTok{end =} \NormalTok{-}\DecValTok{1}\NormalTok{)}
\end{Highlighting}
\end{Shaded}

\begin{verbatim}
## [1] "ring" "tion" "ways" "ting"
\end{verbatim}

\begin{Shaded}
\begin{Highlighting}[]
\CommentTok{#String replacement using str_sub}
\KeywordTok{str_sub}\NormalTok{(y,}\DataTypeTok{start=}\NormalTok{-}\DecValTok{4}\NormalTok{,}\DataTypeTok{end=}\NormalTok{-}\DecValTok{1}\NormalTok{) <-}\StringTok{ 'RING'}
\NormalTok{y}
\end{Highlighting}
\end{Shaded}

\begin{verbatim}
## [1] "stRING"       "manipulaRING" "alRING"       "fascinaRING"
\end{verbatim}

\subsection{String trimming and
padding}\label{string-trimming-and-padding}

One of the major challenges of string parsing is handline additional
whitespaces in words. Often additional widespaces are present on the
left, right or both sides of the word. The \textbf{str\_trim} function
offers an effective way to get rid of these whitespaces.

\begin{Shaded}
\begin{Highlighting}[]
\NormalTok{whitespace.vector <-}\StringTok{ }\KeywordTok{c}\NormalTok{(}\StringTok{'  abc'}\NormalTok{,}\StringTok{'def   '}\NormalTok{,}\StringTok{'     ghi       '}\NormalTok{)}
\CommentTok{#trimming on left sides}
\KeywordTok{str_trim}\NormalTok{(whitespace.vector,}\DataTypeTok{side =} \StringTok{'left'}\NormalTok{)}
\end{Highlighting}
\end{Shaded}

\begin{verbatim}
## [1] "abc"        "def   "     "ghi       "
\end{verbatim}

\begin{Shaded}
\begin{Highlighting}[]
\CommentTok{#trimming on right sides}
\KeywordTok{str_trim}\NormalTok{(whitespace.vector,}\DataTypeTok{side =} \StringTok{'right'}\NormalTok{)}
\end{Highlighting}
\end{Shaded}

\begin{verbatim}
## [1] "  abc"    "def"      "     ghi"
\end{verbatim}

\begin{Shaded}
\begin{Highlighting}[]
\CommentTok{#trimming on both sides}
\KeywordTok{str_trim}\NormalTok{(whitespace.vector,}\DataTypeTok{side =} \StringTok{'both'}\NormalTok{)}
\end{Highlighting}
\end{Shaded}

\begin{verbatim}
## [1] "abc" "def" "ghi"
\end{verbatim}

Conversely we could also pad a string with additional characters for a
defined width using the \textbf{str\_pad()} function. The default
padding character is a space.

\begin{Shaded}
\begin{Highlighting}[]
\CommentTok{#Left padding}
\KeywordTok{str_pad}\NormalTok{(}\StringTok{'abc'}\NormalTok{,}\DataTypeTok{width=}\DecValTok{7}\NormalTok{,}\DataTypeTok{side=}\StringTok{"left"}\NormalTok{)}
\end{Highlighting}
\end{Shaded}

\begin{verbatim}
## [1] "    abc"
\end{verbatim}

\begin{Shaded}
\begin{Highlighting}[]
\CommentTok{#Right padding}
\KeywordTok{str_pad}\NormalTok{(}\StringTok{'abc'}\NormalTok{,}\DataTypeTok{width=}\DecValTok{7}\NormalTok{,}\DataTypeTok{side=}\StringTok{"right"}\NormalTok{)}
\end{Highlighting}
\end{Shaded}

\begin{verbatim}
## [1] "abc    "
\end{verbatim}

\begin{Shaded}
\begin{Highlighting}[]
\CommentTok{#Padding other characters}
\KeywordTok{str_pad}\NormalTok{(}\StringTok{'abc'}\NormalTok{,}\DataTypeTok{width=}\DecValTok{7}\NormalTok{,}\DataTypeTok{side=}\StringTok{"both"}\NormalTok{,}\DataTypeTok{pad=}\StringTok{"#"}\NormalTok{)}
\end{Highlighting}
\end{Shaded}

\begin{verbatim}
## [1] "##abc##"
\end{verbatim}

\subsection{String Wrapping}\label{string-wrapping}

Sometimes text have to be manipulated to neat paragraphs of defined
width. The \textbf{str\_wrap()} function could be used to format the
text into defined paragraphs of specific width.

\begin{Shaded}
\begin{Highlighting}[]
\NormalTok{some_text =}\StringTok{ 'All the Worlds a stage, All men are merely players'}
\KeywordTok{cat}\NormalTok{(}\KeywordTok{str_wrap}\NormalTok{(some_text,}\DataTypeTok{width=}\DecValTok{25}\NormalTok{))}
\end{Highlighting}
\end{Shaded}

\begin{verbatim}
## All the Worlds a stage,
## All men are merely
## players
\end{verbatim}

\subsection{Extracting Words}\label{extracting-words}

Let us complete this chapter with the simple \textbf{word()} function
which extract words from a sentence. We specify the positions of the
word to be extracted from the setence. The default separator value is
space.

\begin{Shaded}
\begin{Highlighting}[]
\CommentTok{#Extracting the first two words of a character vector}
\NormalTok{some.text <-}\StringTok{ }\KeywordTok{c}\NormalTok{(}\StringTok{'The quick brown fox'}\NormalTok{,}\StringTok{'jumps on the brown dog'}\NormalTok{)}
\KeywordTok{word}\NormalTok{(some.text,}\DataTypeTok{start =} \DecValTok{1}\NormalTok{,}\DataTypeTok{end=}\DecValTok{2}\NormalTok{)}
\end{Highlighting}
\end{Shaded}

\begin{verbatim}
## [1] "The quick" "jumps on"
\end{verbatim}

\begin{Shaded}
\begin{Highlighting}[]
\CommentTok{#extracting all but the last word}
\KeywordTok{word}\NormalTok{(some.text,}\DataTypeTok{start=}\DecValTok{1}\NormalTok{,}\DataTypeTok{end=}\NormalTok{-}\DecValTok{2}\NormalTok{)}
\end{Highlighting}
\end{Shaded}

\begin{verbatim}
## [1] "The quick brown"    "jumps on the brown"
\end{verbatim}

\end{document}
